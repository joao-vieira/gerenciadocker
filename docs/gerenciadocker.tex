% ---------------------------------------------------------------------------------------------------------------
% TEMPLATE EM LATEX PARA TRABALHO DE CONCLUSÃO DE CURSO DA URI ERECHIM
% (ESTE TEMPLATE NÃO É UM PROJETO OFICIAL DA URI) CONSULTE SEU ORIETADOR CASO QUEIRA UTILIZA-LO
% Este template foi baseado no template da Universidade Tecnológica Federal do Paraná UTFPR
%
% Template baseado no projeto: http://tcc.tsi.gp.utfpr.edu.br/paginas/modelos-latex-da-utfpr
%
%----------------------------------------------------------------------------------------------------------------
% Codificação: UTF-8
% LaTeX:  abnTeX2
% ---------------------------------------------------------------------------------------------------------------

% CARREGA CLASSE PERSONALIZADA COM AS NORMAS DA URI-----------------------------------------------------------
\documentclass[oneside]{configs/uri-abntex2} %oneside -> impressão apenas frente

% INCLUI ARQUIVOS DE CONFIGURAÇÕES-------------------------------------------------------------------------------
% REFERÊNCIAS------------------------------------------------------------------
\usepackage[%
    alf,
    abnt-emphasize=bf,
    bibjustif,
    recuo=0cm,
    abnt-url-package=url,       % Utiliza o pacote url
    abnt-refinfo=yes,           % Utiliza o estilo bibliográfico abnt-refinfo
    abnt-etal-cite=3,
    abnt-etal-list=3,
    abnt-thesis-year=final
]{abntex2cite}                  % Configura as citações bibliográficas conforme a norma ABNT

% PACOTES----------------------------------------------------------------------
\usepackage[utf8]{inputenc}                                 % Codificação do documento
\usepackage[T1]{fontenc}                                    % Seleção de código de fonte
\usepackage{booktabs}                                       % Réguas horizontais em tabelas
\usepackage{color, colortbl}                                % Controle das cores
\usepackage{float}                                          % Necessário para tabelas/figuras em ambiente multi-colunas
\usepackage{graphicx}                                       % Inclusão de gráficos e figuras
\usepackage{icomma}                                         % Uso de vírgulas em expressões matemáticas
\usepackage{indentfirst}                                    % Indenta o primeiro parágrafo de cada seção
\usepackage{microtype}                                      % Melhora a justificação do documento
\usepackage{multirow, array}                                % Permite tabelas com múltiplas linhas e colunas
\usepackage{subeqnarray}                                    % Permite subnumeração de equações
\usepackage{lastpage}                                       % Para encontrar última página do documento
\usepackage{verbatim}                                       % Permite apresentar texto tal como escrito no documento, ainda que sejam comandos Latex
\usepackage{amsfonts, amssymb, amsmath}                     % Fontes e símbolos matemáticos
%\usepackage[algoruled, portuguese]{algorithm2e}             % Permite escrever algoritmos em português
%\usepackage[scaled]{helvet}                                % Usa a fonte Helvetica
\usepackage{times}                                          % Usa a fonte Times
%\usepackage{palatino}                                      % Usa a fonte Palatino
%\usepackage{lmodern}                                       % Usa a fonte Latin Modern
% \usepackage[bottom]{footmisc}                               % Mantém as notas de rodapé sempre na mesma posição
\usepackage{ae, aecompl}                                    % Fontes de alta qualidade
\usepackage{latexsym}                                       % Símbolos matemáticos
\usepackage{lscape}                                         % Permite páginas em modo "paisagem"
%\usepackage{picinpar}                                      % Dispor imagens em parágrafos
%\usepackage{scalefnt}
\usepackage{setspace}                                    % Permite redimensionar tamanho da fonte
%\usepackage{subfig}                                        % Posicionamento de figuras
%\usepackage{upgreek}                                       % Fonte letras gregas
\usepackage{listings}
\usepackage{nameref}                                        % Permite que o texto seja exibido ao invés do número numa referência (Ex.: \nameref{chap:introducao})

% Redefine a fonte para uma fonte similar a Arial (fonte Helvetica)
% \renewcommand*\familydefault{\sfdefault}
% Configura todo o documento para times new roman
\renewcommand{\familydefault}{ptm}

% CONFIGURAÇÕES DE APARÊNCIA DO PDF FINAL--------------------------------------
\makeatletter
\hypersetup{%
    portuguese,
    colorlinks=true,   % true: "links" coloridos; false: "links" em caixas de texto
    linkcolor=black,    % Define cor dos "links" internos
    citecolor=black,    % Define cor dos "links" para as referências bibliográficas
    filecolor=black,    % Define cor dos "links" para arquivos
    urlcolor=black,     % Define a cor dos "hiperlinks"
    breaklinks=true,
    pdftitle={\@title},
    pdfauthor={\@author},
    pdfkeywords={abnt, latex, abntex, abntex2}
}
\makeatother

% ALTERA O ASPECTO DA COR AZUL--------------------------------------------------
\definecolor{blue}{RGB}{41,5,195}

% REDEFINIÇÃO DE LABELS---------------------------------------------------------
\renewcommand{\algorithmautorefname}{Algoritmo}
\def\equationautorefname~#1\null{Equa\c c\~ao~(#1)\null}

% CRIA ÍNDICE REMISSIVO---------------------------------------------------------
\makeindex

% HIFENIZAÇÃO DE PALAVRAS QUE NÃO ESTÃO NO DICIONÁRIO---------------------------
\hyphenation{%
    qua-dros-cha-ve
    Kat-sa-gge-los
}

%%-----------------------------------------------------------------------------
%% VARIÁVEIS
%%-----------------------------------------------------------------------------
%% Utilize este arquivo para colocar valores que serão útes durante todo o
%%  documento (tal como a versão de um sistema ou o nome de uma ferramenta).

% Adaptive-DSD (singular, itálico)
\newcommand{\adaptive}{\textit{Adaptive-DSD} }

% Contêiner (singular, itálico)
\newcommand{\conteiner}{\textit{contêiner}}

% Containers (plural, itálico)
\newcommand{\containers}{\textit{containers}}

% Docker (itálico)
\newcommand{\docker}{\textit{docker}}

% Docker Network (itálico)
\newcommand{\dockerNetwork}{\textit{docker network}}

% GIT (itálico)
\newcommand{\git}{\textit{GIT}}

% Hangouts (itálico)
\newcommand{\hangouts}{\textit{Hangouts}}

% On-line (itálico)
\newcommand{\online}{\textit{on-line}}

% Software (itálico)
\newcommand{\software}{\textit{software}}

% Trello (itálico)
\newcommand{\trello}{\textit{Trello}}

% Web (itálico)
\newcommand{\web}{\textit{web}}





%%-----------------------------------------------------------------------------
%% COMANDOS
%%-----------------------------------------------------------------------------
%% Os comandos abaixo foram úteis em alguma situação e, para facilitar, foram
%%  colocados neste arquivo para centralizar sua definição.

% Insere aspas duplas acerca de um texto qualquer
\newcommand{\aspas}[1]{``#1''}

% Python style for highlighting
\newcommand\pythonstyle{\lstset{
    language=Python,
    basicstyle=\footnotesize,
    numbers=left,                   
    numberstyle=\tiny\color{gray},  
    stepnumber=1,                             
    numbersep=5pt,                  
    backgroundcolor=\color{white},    
    showspaces=false,               
    showstringspaces=false,         
    showtabs=false,                 
    frame=single,                   
    rulecolor=\color{black},        
    tabsize=2,                      
    captionpos=b,                   
    breaklines=true,                
    breakatwhitespace=false,
    keywordstyle=\color{blue},          
    commentstyle=\color{dkgreen},       
    stringstyle=\color{mauve}          
}}

% Python environment
\lstnewenvironment{python}[1][]
{
\pythonstyle
\lstset{#1}
}
{}

%%-----------------------------------------------------------------------------
%% CORES
%%-----------------------------------------------------------------------------
%% Cores definidas para apresentação de código fonte
\definecolor{dkgreen}{rgb}{0,0.6,0}
\definecolor{gray}{rgb}{0.5,0.5,0.5}
\definecolor{mauve}{rgb}{0.58,0,0.82}


% INCLUI ARQUIVOS DO DESENVOLVIMENTO DO DOCUMENTO (PRÉ-TEXTUAIS, TEXTUAIS, PÓS-TEXTUAIS)-----------------------

% INSERE CAPA
% CAPA---------------------------------------------------------------------------------------------------

% ORIENTAÇÕES GERAIS-------------------------------------------------------------------------------------
% Caso algum dos campos não se aplique ao seu trabalho, como por exemplo,
% se não houve coorientador, apenas deixe vazio.
% Exemplos: 
% \coorientador{}
% \departamento{}

% DADOS DO TRABALHO--------------------------------------------------------------------------------------
% \titulo{\textit{SkillBoard}: \\Sistema para candidatos encontrarem vagas e empresas gerenciarem processos de seleção}
\titulo{\textit{GerenciaDocker}: \\Sistema para gerenciar contêineres}

\titleabstract{Title in English}
\autor{João Vitor Veronese Vieira \\Kelwin Komka \\Vinicius Emanoel Andrade}
\autorcitacao{VIEIRA, João Vitor Veronese and KOMKA, Kelwin and ANDRADE, Vinicius Emanoel} % Sobrenome em maiúsculo
\local{ERECHIM - RS}
\data{2019}

% NATUREZA DO TRABALHO-----------------------------------------------------------------------------------
% Opções: 
% - Projeto de Conclusão de Curso (Disciplina de Projeto)
% - Trabalho de Conclusão de Curso (se for Graduação)
% - Dissertação (se for Mestrado)
% - Tese (se for Doutorado)
% - Projeto de Qualificação (se for Mestrado ou Doutorado)
\projeto{Trabalho de Conclusão de Curso}

% TÍTULO ACADÊMICO---------------------------------------------------------------------------------------
% Opções:
% - Bacharel ou Tecnólogo (Se a natureza for Trabalho de Conclusão de Curso)
% - Mestre (Se a natureza for Dissertação)
% - Doutor (Se a natureza for Tese)
% - Mestre ou Doutor (Se a natureza for Projeto de Qualificação)
\tituloAcademico{Bacharel}

% ÁREA DE CONCENTRAÇÃO E LINHA DE PESQUISA---------------------------------------------------------------
% Se a natureza for Trabalho de Conclusão de Curso, deixe ambos os campos vazios
% Se for programa de Pós-graduação, indique a área de concentração e a linha de pesquisa
\areaconcentracao{}
\linhapesquisa{}

% DADOS DA INSTITUIÇÃO-----------------------------------------------------------------------------------
% Se a natureza for Trabalho de Conclusão de Curso, coloque o nome do curso de graduação em "programa"
% Formato para o logo da Instituição: \logoinstituicao{<escala>}{<caminho/nome do arquivo>}
\instituicao{Universidade Regional Integrada do Alto Uruguai e das Missões Campus de Erechim}
\departamento{Departamento de Engenharias e Ciência da Computação}
\programa{Curso de Ciência da Computação}
\disciplina{Laboratório de Desenvolvimento}
%\logoinstituicao{0.2}{resources/figuras/logo-instituicao.png} 

% DADOS DOS ORIENTADORES---------------------------------------------------------------------------------
\orientador{Nome do orientador}
%\orientador[Orientadora:]{Nome da orientadora}
\instOrientador{Instituição do orientador}

\coorientador{Nome do coorientador}
%\coorientador[Coorientadora:]{Nome da coorientadora}
\instCoorientador{Instituição do coorientador}




% <START>
%   Conteúdo do Documento: base teórica.
% </START>
\begin{document}

    \pretextual
    \imprimircapa                                              	            % Comando para imprimir Capa
    
    % INSERE ELEMENTOS PRÉ-TEXTUAIS
    % SUMÁRIO----------------------------------------------------------------------

\renewcommand{\contentsname}{SUMÁRIO}

\pdfbookmark[0]{\contentsname}{toc}
\tableofcontents*
\cleardoublepage

% OBSERVAÇÕES-------------------------------------------------------------------
% Este arquivo não precisa ser alterado, pois o sumário é gerado automaticamente.
               			        % Sumário
    % Lista de Figuras----------------------------------------------------------------

\pdfbookmark[0]{\listfigurename}{lof}
\listoffigures*
\cleardoublepage

% OBSERVAÇÕES---------------------------------------------------------------------
% Este arquivo não precisa de ser alterado, pois a lista é gerada automaticamente.
           			        % Lista de Figuras

    \textual
    
    % INSERE ELEMENTOS TEXTUAIS
    % INTRODUÇÃO-------------------------------------------------------------------

\chapter{INTRODUÇÃO}
\label{chap:introducao}

\textbf{Introdução} \textit{aqui}.



\section{Seção 2}
\label{sec:secao_2}

Outro Exemplo.

                		            % Introdução
    % DESCRIÇÃO DO PROBLEMA-------------------------------------------------------------------

\chapter{DESCRIÇÃO DO PROBLEMA}
\label{chap:descricao_do_problema}

Em virtude da dinamicidade e agilidade necessárias em tarefas comuns para uma empresa de \software{}, tal como a disponibilização de aplicações para clientes ou mesmo a configuração de ambiente para novos colaboradores na equipe de desenvolvimento, criou-se o conceito técnico de \textit{conteinerização}. De modo resumido, esse conceito pode ser descrito, segundo \cite{Fernandes18}, como \aspas{o processo de distribuir uma aplicação de \software{} de maneira compartimentada, portátil e autossuficiente}. Isto é, uma forma de criar um ambiente completo de qualquer aplicação desenvolvida e \aspas{empacotá-lo}, para posteriormente distribuí-lo e utilizá-lo. 

Dentro desse cenário e tendo em vista os diversos benefícios que essa prática traz aos seus utilizadores, diversas ferramentas foram criadas. No entanto, um \software{} em específico acabou destacando-se como o mais utilizado quando se deseja implantar essa tecnologia. O \textbf{\docker{}} permite o gerenciamento completo de todos os \containers{} criados e, devido às suas funcionalidades, ganhou notoriedade na comunidade. 

No entanto, a utilização diária dessa ferramenta, geralmente realizada através de um terminal, pode se tornar uma tarefa desnecessária e até mesmo complicada, principalmente para um profissional iniciante, pois seu ambiente pode conter diversas especificidades (tal como vários \containers{} executando em paralelo) que, quando se está aprendendo a utilizar a ferramenta, podem ser difíceis de serem implementadas.

Além disso, vale ressaltar que, se necessário, o usuário deve controlar a rede (\dockerNetwork{}) em que os \containers{} estão sendo executados, o que acaba gerando ainda mais dificuldades. Com isso, fica nítido que, apesar de ser um recurso que proporciona inúmeras vantagens aos usuários, ainda existe uma barreira de adoção à essa tecnologia, principalmente em um contexto organizacional, pois o profissional que se dispõe a aprender essa nova ferramenta, precisará conciliar esse aprendizado com a realização de todas as demais atividades tradicionais que ele é encarregado.





% JUSTIFICATIVA-------------------------------------------------------------------
\section{Justificativa}
\label{sec:justificativa}

Baseando-se nos motivos descritos no capítulo \ref{chap:descricao_do_problema} e com o desejo de aprender mais sobre essa moderna ferramenta, o grupo considerou que um monitor \web{} que abstraísse essas dificuldades de gerenciamento em uma interface intuitiva e amigável para o usuário seria, além de uma ferramenta útil para os profissionais que se enquadram nessa situação, um bom assunto para ser o tema deste projeto.          		            % Descrição do Problema
    % OBJETIVOS-------------------------------------------------------------------

\chapter{OBJETIVOS}
\label{chap:objetivos}




\section{Objetivo Geral}
\label{sec:objetivo_geral}

\begin{itemize}
    \item Criar uma ferramenta que possibilite gerenciar os \containers{} em execução na máquina
\end{itemize}




\section{Objetivos Específicos}
\label{sec:objetivos_especificos}

\begin{itemize}
    \item Tornar a ferramenta flexível, permitindo que o usuário escolha o sistema operacional do \conteiner{} (com 4 opções)
    \item Construir uma interface amigável e intuitiva para o usuário
    \item Executar corretamente o algoritmo (\adaptive{}), que detectará falha nos \containers{}
\end{itemize}                  		            % Objetivos
    % CONCLUSÃO-------------------------------------------------------------------
\chapter{CONCLUSÃO}
\label{chap:conclusao}


% Desenvolver um sistema é um trabalho que vai muito além de programá-lo, e dentre todas as suas etapas, nós daremos enfâse a uma em específico, a arquitetura do sistema

% Um desenvolvedor deve ter habilidades que vão além da programação, dominar múltiplas áreas de conhecimento é praticamente obrigatório, pois é necessário para ter controle de suas soluções, até por que uma solução (dependendo de sua complexidade) exige que inúmeras tecnologias e conhecimentos sejam interligados, que no final de fato solucionarão um problema.

% Esse tipo de exercício exige dos alunos conhecimentos que serão necessários no mercado de trabalho, pois um desenvolvedor deve ser versátil, eficiente e tem a obrigação de dominar sua soluções.

% O trabalho possui uma proposta muito interessante, pois ele faz com que o aluno pense em um problema real e busque solucioná-lo. É notório que solucionar um problema exige que múltiplas tecnologias e áreas de conhecimento sejam interligadas, até porque criar uma solução vai muito além de programá-la.

% Existe uma grande aproximação desse exercício acadêmico com o que o mercado de trabalho exige, principalmente pelo fato de estarmos tentando solucionar um problema real. Vale ressaltar que esse realismo foi um grande motivador durante a realização do projeto.

O trabalho possui uma proposta muito interessante, pois ele faz com que o aluno pense em um problema real e busque solucioná-lo. Com base nisso e também na notoriedade de que solucionar um problema exige que os alunos interliguem múltiplas tecnologias e áreas de conhecimento, ele provou-se muito relevante para todos os membros do grupo.

Notou-se uma grande aproximação desse exercício acadêmico com o que o mercado de trabalho exige, principalmente pelo fato de induzir os alunos a solucionar um problema real, o que mostrou-se muito benéfico para os membros do grupo, pois acabou motivando a todos. Ainda dentro desse paralelismo com o mercado de trabalho, o maior benefício dentre os pontos positivos, pode-se dizer, é aprender a criar uma solução, passando por todas as etapas de construção da mesma e adotando métodos para que as entregas e a distribuição de tarefas sejam organizadas.

Outro ponto muito importante que deve ser mencionado, é a possibilidade de pôr em prática muito do que foi aprendido em outras disciplinas e não havia sido implementado. Vale destacar, novamente, que isso só é possível devido à proposta do trabalho, onde é dada a liberdade para que o aluno escolha quais serão as áreas que ele deseja utilizar. Inclusive, esse ponto pode ser muito bem aproveitado, pois incentiva que o aluno aprofunde-se ainda mais em alguma das áreas da Ciência da Computação, permitindo uma enorme extração de ideias para o Trabalho de Conclusão de Curso ou até mesmo para áreas de atuação que ele possa trabalhar.

A maior dificuldade do grupo, no entanto, foi cumprir o cronograma esboçado na proposta entregue ao professor, sendo que alguns prazos acabaram não sendo respeitados. Todos esses atrasos ocorreram influenciados por fatores externos como, por exemplo, tarefas que tiveram que ter uma prioridade maior naquele período de tempo. Porém, é importante enfatizar que o cronograma foi importante para a organização de tarefas, pois ele proporcionou uma visão geral do projeto que foi crucial na distribuição das mesmas e, com isso, todas as decisões permearam o cronograma inicialmente montado.

Correndo o risco de ser redundante, gostaríamos de fortalecer que a proposta do trabalho é extremamente válida e de fato agrega muito conhecimento para os alunos, como anteriormente mencionado, pois existe um paralelismo muito interessante com as exigências do mercado de trabalho atual. Não existe nenhum tipo de crítica quanto ao formato ou modelo propostos para sua execução. Por fim, entregamos o projeto com uma sensação de dever cumprido, pois atingimos todos os objetivos que foram inicialmente definidos com a ajuda do professor.                   		            % Conclusão

    \postextual

    % INSERE ELEMENTOS PÓS-TEXTUAIS
    % REFERÊNCIAS------------------------------------------------------------------

% Carrega o arquivo "referencias.bib" e extrai automaticamente as referências citadas
{\renewcommand{\contentsname}{MAC}}
\bibliography{./gerenciadocker}
\bibliographystyle{abntex2-alf} % Define o estilo ABNT para formatar a lista de referências
% OBSERVAÇÕES------------------------------------------------------------------
% Este arquivo não precisa ser alterado.           			        % Referências


\end{document}
% <ENCERRA>
%   Conteúdo do Documento: base teórica.
% </ENCERRA>
%%-----------------------------------------------------------------------------
%% VARIÁVEIS
%%-----------------------------------------------------------------------------
%% Utilize este arquivo para colocar valores que serão útes durante todo o
%%  documento (tal como a versão de um sistema ou o nome de uma ferramenta).

% Adaptive-DSD (singular, itálico)
\newcommand{\adaptive}{\textit{Adaptive-DSD} }

% Contêiner (singular, itálico)
\newcommand{\conteiner}{\textit{contêiner}}

% Containers (plural, itálico)
\newcommand{\containers}{\textit{containers}}

% Docker (itálico)
\newcommand{\docker}{\textit{docker}}

% Docker Network (itálico)
\newcommand{\dockerNetwork}{\textit{docker network}}

% GIT (itálico)
\newcommand{\git}{\textit{GIT}}

% Hangouts (itálico)
\newcommand{\hangouts}{\textit{Hangouts}}

% On-line (itálico)
\newcommand{\online}{\textit{on-line}}

% Software (itálico)
\newcommand{\software}{\textit{software}}

% Trello (itálico)
\newcommand{\trello}{\textit{Trello}}

% Web (itálico)
\newcommand{\web}{\textit{web}}





%%-----------------------------------------------------------------------------
%% COMANDOS
%%-----------------------------------------------------------------------------
%% Os comandos abaixo foram úteis em alguma situação e, para facilitar, foram
%%  colocados neste arquivo para centralizar sua definição.

% Insere aspas duplas acerca de um texto qualquer
\newcommand{\aspas}[1]{``#1''}

% Python style for highlighting
\newcommand\pythonstyle{\lstset{
    language=Python,
    basicstyle=\footnotesize,
    numbers=left,                   
    numberstyle=\tiny\color{gray},  
    stepnumber=1,                             
    numbersep=5pt,                  
    backgroundcolor=\color{white},    
    showspaces=false,               
    showstringspaces=false,         
    showtabs=false,                 
    frame=single,                   
    rulecolor=\color{black},        
    tabsize=2,                      
    captionpos=b,                   
    breaklines=true,                
    breakatwhitespace=false,
    keywordstyle=\color{blue},          
    commentstyle=\color{dkgreen},       
    stringstyle=\color{mauve}          
}}

% Python environment
\lstnewenvironment{python}[1][]
{
\pythonstyle
\lstset{#1}
}
{}

%%-----------------------------------------------------------------------------
%% CORES
%%-----------------------------------------------------------------------------
%% Cores definidas para apresentação de código fonte
\definecolor{dkgreen}{rgb}{0,0.6,0}
\definecolor{gray}{rgb}{0.5,0.5,0.5}
\definecolor{mauve}{rgb}{0.58,0,0.82}
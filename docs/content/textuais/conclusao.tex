% CONCLUSÃO - SkillBoard -------------------------------------------------------------------
\chapter{CONCLUSÃO}
\label{chap:conclusao}


% Desenvolver um sistema é um trabalho que vai muito além de programá-lo, e dentre todas as suas etapas, nós daremos enfâse a uma em específico, a arquitetura do sistema

% Um desenvolvedor deve ter habilidades que vão além da programação, dominar múltiplas áreas de conhecimento é praticamente obrigatório, pois é necessário para ter controle de suas soluções, até por que uma solução (dependendo de sua complexidade) exige que inúmeras tecnologias e conhecimentos sejam interligados, que no final de fato solucionarão um problema.

% Esse tipo de exercício exige dos alunos conhecimentos que serão necessários no mercado de trabalho, pois um desenvolvedor deve ser versátil, eficiente e tem a obrigação de dominar sua soluções.

O trabalho possui uma proposta muito interessante, pois ele faz com que o aluno pense em um problema real e busque solucioná-lo, e sabemos que solucionar um problema exige que múltiplas tecnologias e áreas de conhecimento sejam interligadas, até porque criar uma solução vai muito além de programá-la.

Notamos uma grande aproximação desse exercício acadêmico com o que o mercado de trabalho exige, principalmente pelo fato de estarmos tentando solucionar um problema real, isso foi muito benéfico para os membros do grupo, pois acabou motivando a todos. Ainda dentro desse paralelismo com o mercado de trabalho, temos o maior dos ganhos, que é aprender a criar uma solução, passando por todas as etapas e adotando métodos para que as entregas e a distribuição de tarefas mantenham-se organizadas.

Outro ponto muito importante que deve ser mencionado, é o de termos a possibilidade de por em prática muito do que foi aprendido e não havia sido implementado, vale ressaltar, isso só é possível devido à proposta do trabalho que da liberdade para que o aluno escolha quais serão as áreas que ele deseja utilizar. Inclusive, esse ponto pode ser muito bem aproveitado, pois permite que o aluno aprofunde-se ainda mais em alguma das áreas da Ciência da Computação e isso é importante para iniciar decisões e idéias sobre o Trabalho de Conclusão de Curso.

Nossa maior dificuldade foi seguir o cronograma esboçado na proposta entregue ao professor, muitos dos prazos acabaram não sendo respeitados, todos eles influenciados por fatores externos, como por exemplo, tarefas que tiveram que ter uma prioridade maior naquele período de tempo, porém, vale a ressalva, o cronograma foi importante para nossa organização de tarefas, ele proporcionou uma visão geral que foi crucial na distribuição das mesmas, todas as decisões permearam o cronograma inicialmente montado.

Gostaríamos de deixar explícito que a proposta do trabalho é extremamente válida e de fato agrega muito conhecimento para os alunos, como anteriormente mencionado, existe um paralelismo muito interessante com o que o mercado de trabalho exige de um profissional. Não existe nenhum tipo de crítica quanto ao formato ou modelo do trabalho, essa flexibilidade facilitou bastante a criação do mesmo. Por fim, deixamos nossa única sugestão, adotar obrigatoriamente uma metodologia real no desenvolvimento, dessa forma os alunos que não tiverem acostumados com o meio corporativo, possam ir se habituando.
% CONCLUSÃO-------------------------------------------------------------------
\chapter{CONCLUSÃO}
\label{chap:conclusao}


% Desenvolver um sistema é um trabalho que vai muito além de programá-lo, e dentre todas as suas etapas, nós daremos enfâse a uma em específico, a arquitetura do sistema

% Um desenvolvedor deve ter habilidades que vão além da programação, dominar múltiplas áreas de conhecimento é praticamente obrigatório, pois é necessário para ter controle de suas soluções, até por que uma solução (dependendo de sua complexidade) exige que inúmeras tecnologias e conhecimentos sejam interligados, que no final de fato solucionarão um problema.

% Esse tipo de exercício exige dos alunos conhecimentos que serão necessários no mercado de trabalho, pois um desenvolvedor deve ser versátil, eficiente e tem a obrigação de dominar sua soluções.

% O trabalho possui uma proposta muito interessante, pois ele faz com que o aluno pense em um problema real e busque solucioná-lo. É notório que solucionar um problema exige que múltiplas tecnologias e áreas de conhecimento sejam interligadas, até porque criar uma solução vai muito além de programá-la.

% Existe uma grande aproximação desse exercício acadêmico com o que o mercado de trabalho exige, principalmente pelo fato de estarmos tentando solucionar um problema real. Vale ressaltar que esse realismo foi um grande motivador durante a realização do projeto.

O trabalho possui uma proposta muito interessante, pois ele faz com que o aluno pense em um problema real e busque solucioná-lo. Com base nisso e também na notoriedade de que solucionar um problema exige que os alunos interliguem múltiplas tecnologias e áreas de conhecimento, ele provou-se muito relevante para todos os membros do grupo.

Notou-se uma grande aproximação desse exercício acadêmico com o que o mercado de trabalho exige, principalmente pelo fato de induzir os alunos a solucionar um problema real, o que mostrou-se muito benéfico para os membros do grupo, pois acabou motivando a todos. Ainda dentro desse paralelismo com o mercado de trabalho, o maior benefício dentre os pontos positivos, pode-se dizer, é aprender a criar uma solução, passando por todas as etapas de construção da mesma e adotando métodos para que as entregas e a distribuição de tarefas sejam organizadas.

Outro ponto muito importante que deve ser mencionado, é a possibilidade de pôr em prática muito do que foi aprendido em outras disciplinas e não havia sido implementado. Vale destacar, novamente, que isso só é possível devido à proposta do trabalho, onde é dada a liberdade para que o aluno escolha quais serão as áreas que ele deseja utilizar. Inclusive, esse ponto pode ser muito bem aproveitado, pois incentiva que o aluno aprofunde-se ainda mais em alguma das áreas da Ciência da Computação, permitindo uma enorme extração de ideias para o Trabalho de Conclusão de Curso ou até mesmo para áreas de atuação que ele possa trabalhar.

A maior dificuldade do grupo, no entanto, foi cumprir o cronograma esboçado na proposta entregue ao professor, sendo que alguns prazos acabaram não sendo respeitados. Todos esses atrasos ocorreram influenciados por fatores externos como, por exemplo, tarefas que tiveram que ter uma prioridade maior naquele período de tempo. Porém, é importante enfatizar que o cronograma foi importante para a organização de tarefas, pois ele proporcionou uma visão geral do projeto que foi crucial na distribuição das mesmas e, com isso, todas as decisões permearam o cronograma inicialmente montado.

Correndo o risco de ser redundante, gostaríamos de fortalecer que a proposta do trabalho é extremamente válida e de fato agrega muito conhecimento para os alunos, como anteriormente mencionado, pois existe um paralelismo muito interessante com as exigências do mercado de trabalho atual. Não existe nenhum tipo de crítica quanto ao formato ou modelo propostos para sua execução. Por fim, entregamos o projeto com uma sensação de dever cumprido, pois atingimos todos os objetivos que foram inicialmente definidos com a ajuda do professor.
% METODOLOGIA-------------------------------------------------------------------

\chapter{METODOLOGIA}
\label{chap:metodologia}

Sempre que almeja-se alcançar um objetivo ao final de uma tarefa, trabalho ou projeto, naturalmente o indíviduo ou equipe responsável por essa demanda irá adotar certas práticas que, no seu modo de ver, irão lhe ajudar a chegar mais assertivamente no resultado esperado. Essas práticas nada mais são do que a metodologia que será adotada para a execução da sua incumbência, isto é, \cite{Stoodi18} \aspas{a sistematização utilizada para alcançar um resultado}.

Quando um trabalho é consideravelmente grande (tal como esse projeto, por exemplo), provavelmente ele será executado por várias pessoas e, naturalmente, quanto maior for a equipe, mais difícil será organizar as tarefas de cada integrante, bem como suas responsabilidades. Em virtude dessa complexidade de controle, ao longo do tempo, diversas metodologias de gerenciamento de projetos foram criadas, tais como: XP, Scrum, Kanban, Price2, Agile, entre outras, com o objetivo de auxiliar empresas e equipes profissionais a controlarem melhor suas demandas e entregas.

Apesar deste projeto ser um tanto complexo e a equipe realizadora do mesmo ser formada por três integrantes, nenhuma dessas metodologias foi implantada do início ao fim pois, no ponto de vista dos integrantes, seria um controle exagerado e perderia-se mais tempo seguindo as práticas indicadas do que realizando as tarefas propostas. Porém, isso não quer dizer que não houve organização no trabalho - pelo contrário -, quer dizer apenas que utilizou-se somente aquilo que seria útil para a equipe e que realmente seria implantado.

Um exemplo claro dessa \aspas{utilização sob demanda} é a \textit{definição de prazos}. O Scrum, uma das metodologias citadas acima e bastante utilizada no ramo de desenvolvimento de \software{}, por exemplo, tem como uma das suas principais características o \textit{sprint}, que é \cite{Junior17} \aspas{uma forma de facilitar a divisão de um projeto em etapas ao longo do tempo}. Ou seja, sua principal função é definir entregas esperadas ao longo da duração do projeto e, apesar de não utilizar o Scrum em sua essência, o grupo definiu prazos para as tarefas desde o início do projeto e procurou respeitá-las sempre que possível.

Da mesma forma, diversas outras práticas recomendadas por metodologias consagradas de gerenciamento de projetos foram utilizadas pelo grupo para que fosse possível chegar à versão final dessa solução. Alguns exemplos de práticas adotadas são: reuniões remotas através do \software{} \hangouts{}; reuniões presenciais para definição de itens; organização das tarefas e atribuição de responsáveis utilizando o \software{} \trello{}; versionamento do código da aplicação com o \git{} e criação do documento final utilizando \LaTeX{}.

Portanto, adotando essas práticas sempre que necessário e procurando respeitar os prazos definidos, a equipe conseguiu auto-orgarnizar-se e entregar suas obrigações sem maiores problemas. Além disso, é possível afirmar que a experiência foi agradável e não houve nenhum tipo de problema com gestão das tarefas para cada integrante, o que possibilitou atingir um resultado que atendeu às expectativas iniciais.
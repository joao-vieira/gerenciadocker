% DESCRIÇÃO DA IMPLEMENTAÇÃO-------------------------------------------------------------------

\chapter{DESCRIÇÃO DA IMPLEMENTAÇÃO}
\label{chap:descricao_da_implementacao}

% ADAPTIVE DSD-------------------------------------------------------------------
\section{Adaptive DSD}
\label{sec:adaptiveDSD}

O \adaptive (\textit{Adaptive Distributed System-Level Diagnosis}) é um algoritmo para diagnóstico em redes completamente conectadas. Onde seu funcionamento é, ao mesmo tempo 
adaptativo e distribuído. Foi desenvolvido para que cada máquina que possua o algoritmo em execução possa realizar o teste e também ser testada por outras máquinas na rede.
É caracterizado como adaptativo por não depender e nem restrigir o número de máquinas na rede, necessitando, no mínimo uma máquina para o teste. Para a execução dos testes, não é levado
em consideração falhas na rede, pois o objetivo deste algoritmo é, testar o processamento ou funcionamento específico de um processo na máquina.

\subsection{Funcionamento}
\label{sub:adaptiveDSD_Funcionamento}
O algoritmo possui duas listas, que possuem de tamanho o número de máquinas conectadas à rede, as listas são: o vetor TESTED\_UP, que irá guardar na posição da máquina atual, 
o índice da máquina testada que possui funcionamento normal; o vetor STATE, que armazena o estado das máquinas, tendo inicialmente o valor FALHO para todas e, caso uma máquina tenha seu 
funcionamento correto confirmado, esta receberá o valor NORMAL no vetor. A cada rodada os vetores são atualizados e enviados às outras máquinas na rede.

Na primeira rodada, uma máquina irá iniciar o teste seguindo a lista de máquina existentes e disponíveis na rede. Esta máquina irá percorrer a
lista de máquinas e fará uma requisição de teste à próxima máquina da lista. No caso da máquina à ser testada retornar uma resposta de funcionamento correto, a máquina que está realizando o 
teste atualiza os dados e envia à máquina testada, que por sua vez irá executar o mesmo processo com a máquina seguinte, até que todas as máquinas tenham sido testadas. Por outro lado, se 
a máquina testada retornar algum erro, será marcada como falha, e a máquina que está testando irá testar a próxima máquina da lista, até encontrar outro máquina com funcionamento normal 
ou até que a lista de máquinas disponíveis acabe.

A segunda rodada será para atualizar as informações de todas as máquinas na rede sobre o estado de funcionamento de cada máquina. Inicialmente a primeira máquina com funcionamento normal 
irá verificar na lista de máquinas, qual a próxima máquina funcionando e irá enviar os dados da rede. Ao receber os dados da rede, a máquina receptora irá prosseguir com a distrbuição 
de informações.

\subsection{Algoritmo}
\label{sub:adaptiveDSD_Algoritmo}
O algoritmo \adaptive inicia sua execução criando uma conexão em uma porta por um \textit{socket} e fica escutando esta porta, até que uma conexão seja estabelecida por outra máquina.
Ao final de toda requisição realizada por outro máquina, o algoritmo volta a escutar e aguardar uma nova conexão por \textit{socket} com a porta.

\vspace*{1cm}
\begin{python}
    tcp = socket.socket(socket.AF_INET, socket.SOCK_STREAM)
    tupla = (ip_host, int(porta_host))
    tcp.bind(tupla)
    tcp.listen(1)

    conexao, cliente = tcp.accept() 
    ReceberRequisicao(conexao)
\end{python}
\vspace*{1cm}

O método \textbf{ReceberRequisicao} direciona para o fluxo requisitado pela máquina que está conectando com a máquina atual. As seguintes mensagens podem ser enviadas para realizar determinadas ações:

\vspace*{1cm}
\begin{enumerate}
    \item \textbf{'start'}: mensagem enviada pelo gerenciador para realizar uma verificação das máquinas. A máquina será considerada a primeira da lista (posição 0) e terá o estado NORMAL. Ao receber esta mensagem,
    irá executar o método IniciarTeste.
    \item \textbf{'check'}: mensagem enviada pela máquina que está realizando o teste no momento. Ao receber esta mensagem a máquina atual irá realizar uma verificação de funcionamento e, 
    irá retornar se possui falha ou não.
    \item \textbf{'keepTest'}: mensagem enviada pela máquina que está realizando o teste caso a máquina atual esteja com funcionamento NORMAL. Informa a máquina atual para dar continuidade ao teste de funcionamento.
    \item \textbf{'keepInfo'}: mensagem enviada por uma máquina na lista de máquinas com status NORMAL. Informa a máquina atual para manter as informações do teste realizado e prosseguir com a distribução da informação.
    \item \textbf{'info'}: mensagem enviada pelo gerenciador para receber as informações do ultimo teste. A máquina atual irá retornar ao gerenciador o status de cada máquina da rede.
  \end{enumerate}

\vspace*{1cm}
\begin{python}
def ReceberRequisicao(conexao):
    print("Aguardando mensagem...")
    msg = ReceberResposta(conexao)

    if msg == "start":
        IniciarTeste(conexao)
    elif msg == "check":
        RealizarVerificacao(conexao)
    elif msg == "keepTest":
        ContinuarTeste(conexao)
    elif msg == "keepInfo":
        ManterInformacao(conexao)
    elif msg == "info":
        RetornaInformacao(conexao, False)
\end{python}
\vspace*{1cm}

O método \textbf{IniciarTeste} inicia recebendo do gerenciador uma lista das máquinas, contendo o IP e porta, para conexão. Cria as listas TESTE\_UP e STATE e inicia o teste.

\vspace*{1cm}
\begin{python}
def IniciarTeste(conexao):
    global maquinas
    global tested_up
    global state

    EnviarResposta(conexao, "OK")
    json_maquinas = ReceberResposta(conexao)
    maquinas = json.loads(json_maquinas)

    tested_up = ["-1"] * len(maquinas)
    state = ["FALHO"] * len(maquinas)

    print("maquinas: " + str(maquinas))
    print("tested_up: " + str(tested_up))
    print("state: " + str(state))

    index = maquinas.index(ip_host+":"+str(porta_host))
    state[index] = "NORMAL"
    if index == len(maquinas)-1:
        index = 0
    TestarMaquina(index)
\end{python}
\vspace*{1cm}

O método \textbf{TestarMaquina} percorre a lista de máquinas, executa o método \textbf{CriarConexao} para criar a conexão com a máquina a ser testada e envia a mensagem 'check'. Caso 
a máquina testada retornar a confirmação de funcionamento, a máquina atual procede em enviar as informações existentes à máquina testada, primeiramente enviando a mensagem 'keepTest' para informar a outra máquina 
a dar continuidade no teste. Caso a máquina testada retornar erro, esta será marcada com um 'X' no vetor TESTED\_UP e como 'FALHO' no vetor STATE.
Ao chegar na última máquina da lista, o método \textbf{DistribuirInformacao} é executado.

\vspace*{1cm}
\begin{python}
def TestarMaquina(index):
    global maquinas
    global tested_up
    global state
    global ip_host
    global porta_host

    index_maquina = maquinas.index(ip_host+":"+str(porta_host))
    while index < len(maquinas):
        if index_maquina != index and tested_up[index] != "X":
            host, porta = maquinas[index].split(":")
            time.sleep(0.5)
            maquina = CriarConexao(host, porta)

            msg = EnviarInformacao(maquina, "check")
            if msg == "OK":
                time.sleep(0.5)
                maquina = CriarConexao(host, porta)
                EnviarInformacao(maquina, "keepTest")

                tested_up[index_maquina] = str(index)
                state[index] = "NORMAL"

                json_maquinas = json.dumps(maquinas)
                json_tested = json.dumps(tested_up)
                json_state = json.dumps(state)
                
                EnviarInformacao(maquina, json_maquinas)
                EnviarInformacao(maquina, json_tested)
                EnviarInformacao(maquina, json_state)

                maquina.close()
                break
            else:
                print("maquina com falha: " + str(index))
                tested_up[index] = "X"
                state[index] = "FALHO"
                index = index + 1
        else:
            index = index + 1

    if index == len(maquinas):
        print("Iniciar segundo ciclo...")
        DistribuirInformacao()
\end{python}
\vspace*{1cm}

O método \textbf{ContinuarTeste} recebe as informações coletadas até o momento no teste, enviando uma confirmação a cada envio. Após isto, uma verificação é realizada para direcionar 
o teste para próxima máquina da lista, ou para iniciar a distrbuição das informações, caso não haja mais máquinas não testadas na rede.

\vspace*{1cm}
\begin{python}
def ContinuarTeste(conexao):
    global maquinas
    global tested_up
    global state

    EnviarResposta(conexao, "OK")

    json_maquinas = ReceberResposta(conexao)
    maquinas = json.loads(json_maquinas)
    EnviarResposta(conexao, "OK")

    json_tested = ReceberResposta(conexao)
    tested_up = json.loads(json_tested)
    EnviarResposta(conexao, "OK")

    json_state = ReceberResposta(conexao)
    state = json.loads(json_state)
    EnviarResposta(conexao, "OK")

    index = 0
    segundo_ciclo = True
    while index < len(tested_up):
        if tested_up[index] == "-1":
            segundo_ciclo = False
            break
        index = index + 1

    if segundo_ciclo:
        print("Iniciar segundo ciclo...")
        DistribuirInformacao()
    else:
        index_maquina = maquinas.index(ip_host+":"+str(porta_host))
        if index_maquina == len(maquinas)-1:
            index_maquina = 0
        TestarMaquina(index_maquina)
\end{python}
\vspace*{1cm}

O método \textbf{ManterInformacao} recebe as informações do teste finalizado, enviando uma confirmação a cada envio. Por fim, a máquina envia estas informações à próxima máquina com status 
'NORMAL'.

\vspace*{1cm}
\begin{python}
def ManterInformacao(conexao):
    global tested_up
    global state
    global porta_host
    global ip_host
    EnviarResposta(conexao, "OK")

    json_tested = ReceberResposta(conexao)
    tested_up = json.loads(json_tested)
    EnviarResposta(conexao, "OK")

    json_state = ReceberResposta(conexao)
    state = json.loads(json_state)
    EnviarResposta(conexao, "OK")

    index_host = maquinas.index(ip_host+":"+str(porta_host))
    indexMaquina = int(tested_up[index_host])
    if index_host < len(maquinas)-1 and indexMaquina > index_host:
        time.sleep(0.5)
        host, porta = maquinas[int(indexMaquina)].split(":")
        maquina = CriarConexao(host, porta)

        EnviarInformacao(maquina, "keepInfo")
        RetornaInformacao(maquina, True)
        maquina.close()
\end{python}
\vspace*{1cm}

O método \textbf{RetornaInformacao} retorna as informações do teste. Caso o parâmetro \textbf{verificacao} seja verdadeiro, ao enviar uma informação, a máquina irá esperar por uma resposta 
de confirmação, se não, irá apenas enviar as informações, sem esperar por uma resposta de confirmação.

\vspace*{1cm}
\begin{python}
def RetornaInformacao(conexao, verificacao):
    global tested_up
    global state

    json_tested = json.dumps(tested_up)
    json_state = json.dumps(state)

    if verificacao:
        EnviarInformacao(conexao, json_tested)
        EnviarInformacao(conexao, json_state)
    else:
        EnviarResposta(conexao, json_tested)
        EnviarResposta(conexao, json_state)
    conexao.close()
\end{python}